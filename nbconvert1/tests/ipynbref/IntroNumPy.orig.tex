%% This file was auto-generated by IPython.
%% Conversion from the original notebook file:
%% tests/ipynbref/IntroNumPy.orig.ipynb
%%
\documentclass[11pt,english]{article}

%% This is the automatic preamble used by IPython.  Note that it does *not*
%% include a documentclass declaration, that is added at runtime to the overall
%% document.

\usepackage{amsmath}
\usepackage{amssymb}
\usepackage{graphicx}
\usepackage{ucs}
\usepackage[utf8x]{inputenc}

% needed for markdown enumerations to work
\usepackage{enumerate}

% Slightly bigger margins than the latex defaults
\usepackage{geometry}
\geometry{verbose,tmargin=3cm,bmargin=3cm,lmargin=2.5cm,rmargin=2.5cm}

% Define a few colors for use in code, links and cell shading
\usepackage{color}
\definecolor{orange}{cmyk}{0,0.4,0.8,0.2}
\definecolor{darkorange}{rgb}{.71,0.21,0.01}
\definecolor{darkgreen}{rgb}{.12,.54,.11}
\definecolor{myteal}{rgb}{.26, .44, .56}
\definecolor{gray}{gray}{0.45}
\definecolor{lightgray}{gray}{.95}
\definecolor{mediumgray}{gray}{.8}
\definecolor{inputbackground}{rgb}{.95, .95, .85}
\definecolor{outputbackground}{rgb}{.95, .95, .95}
\definecolor{traceback}{rgb}{1, .95, .95}

% Framed environments for code cells (inputs, outputs, errors, ...).  The
% various uses of \unskip (or not) at the end were fine-tuned by hand, so don't
% randomly change them unless you're sure of the effect it will have.
\usepackage{framed}

% remove extraneous vertical space in boxes
\setlength\fboxsep{0pt}

% codecell is the whole input+output set of blocks that a Code cell can
% generate.

% TODO: unfortunately, it seems that using a framed codecell environment breaks
% the ability of the frames inside of it to be broken across pages.  This
% causes at least the problem of having lots of empty space at the bottom of
% pages as new frames are moved to the next page, and if a single frame is too
% long to fit on a page, will completely stop latex from compiling the
% document.  So unless we figure out a solution to this, we'll have to instead
% leave the codecell env. as empty.  I'm keeping the original codecell
% definition here (a thin vertical bar) for reference, in case we find a
% solution to the page break issue.

%% \newenvironment{codecell}{%
%%     \def\FrameCommand{\color{mediumgray} \vrule width 1pt \hspace{5pt}}%
%%    \MakeFramed{\vspace{-0.5em}}}
%%  {\unskip\endMakeFramed}

% For now, make this a no-op...
\newenvironment{codecell}{}

 \newenvironment{codeinput}{%
   \def\FrameCommand{\colorbox{inputbackground}}%
   \MakeFramed{\advance\hsize-\width \FrameRestore}}
 {\unskip\endMakeFramed}

\newenvironment{codeoutput}{%
   \def\FrameCommand{\colorbox{outputbackground}}%
   \vspace{-1.4em}
   \MakeFramed{\advance\hsize-\width \FrameRestore}}
 {\unskip\medskip\endMakeFramed}

\newenvironment{traceback}{%
   \def\FrameCommand{\colorbox{traceback}}%
   \MakeFramed{\advance\hsize-\width \FrameRestore}}
 {\endMakeFramed}

% Use and configure listings package for nicely formatted code
\usepackage{listingsutf8}
\lstset{
  language=python,
  inputencoding=utf8x,
  extendedchars=\true,
  aboveskip=\smallskipamount,
  belowskip=\smallskipamount,
  xleftmargin=2mm,
  breaklines=true,
  basicstyle=\small \ttfamily,
  showstringspaces=false,
  keywordstyle=\color{blue}\bfseries,
  commentstyle=\color{myteal},
  stringstyle=\color{darkgreen},
  identifierstyle=\color{darkorange},
  columns=fullflexible,  % tighter character kerning, like verb
}

% The hyperref package gives us a pdf with properly built
% internal navigation ('pdf bookmarks' for the table of contents,
% internal cross-reference links, web links for URLs, etc.)
\usepackage{hyperref}
\hypersetup{
  breaklinks=true,  % so long urls are correctly broken across lines
  colorlinks=true,
  urlcolor=blue,
  linkcolor=darkorange,
  citecolor=darkgreen,
  }

% hardcode size of all verbatim environments to be a bit smaller
\makeatletter 
\g@addto@macro\@verbatim\small\topsep=0.5em\partopsep=0pt
\makeatother 

% Prevent overflowing lines due to urls and other hard-to-break entities.
\sloppy

\begin{document}

\section{An Introduction to the Scientific Python Ecosystem}
While the Python language is an excellent tool for general-purpose
programming, with a highly readable syntax, rich and powerful data types
(strings, lists, sets, dictionaries, arbitrary length integers, etc) and
a very comprehensive standard library, it was not designed specifically
for mathematical and scientific computing. Neither the language nor its
standard library have facilities for the efficient representation of
multidimensional datasets, tools for linear algebra and general matrix
manipulations (an essential building block of virtually all technical
computing), nor any data visualization facilities.

In particular, Python lists are very flexible containers that can be
nested arbitrarily deep and which can hold any Python object in them,
but they are poorly suited to represent efficiently common mathematical
constructs like vectors and matrices. In contrast, much of our modern
heritage of scientific computing has been built on top of libraries
written in the Fortran language, which has native support for vectors
and matrices as well as a library of mathematical functions that can
efficiently operate on entire arrays at once.

\subsection{Scientific Python: a collaboration of projects built by scientists}
The scientific community has developed a set of related Python libraries
that provide powerful array facilities, linear algebra, numerical
algorithms, data visualization and more. In this appendix, we will
briefly outline the tools most frequently used for this purpose, that
make ``Scientific Python'' something far more powerful than the Python
language alone.

For reasons of space, we can only describe in some detail the central
Numpy library, but below we provide links to the websites of each
project where you can read their documentation in more detail.

First, let's look at an overview of the basic tools that most scientists
use in daily research with Python. The core of this ecosystem is
composed of:

\begin{itemize}
\item
  Numpy: the basic library that most others depend on, it provides a
  powerful array type that can represent multidmensional datasets of
  many different kinds and that supports arithmetic operations. Numpy
  also provides a library of common mathematical functions, basic linear
  algebra, random number generation and Fast Fourier Transforms. Numpy
  can be found at \href{http://numpy.scipy.org}{numpy.scipy.org}
\item
  Scipy: a large collection of numerical algorithms that operate on
  numpy arrays and provide facilities for many common tasks in
  scientific computing, including dense and sparse linear algebra
  support, optimization, special functions, statistics, n-dimensional
  image processing, signal processing and more. Scipy can be found at
  \href{http://scipy.org}{scipy.org}.
\item
  Matplotlib: a data visualization library with a strong focus on
  producing high-quality output, it supports a variety of common
  scientific plot types in two and three dimensions, with precise
  control over the final output and format for publication-quality
  results. Matplotlib can also be controlled interactively allowing
  graphical manipulation of your data (zooming, panning, etc) and can be
  used with most modern user interface toolkits. It can be found at
  \href{http://matplotlib.sf.net}{matplotlib.sf.net}.
\item
  IPython: while not strictly scientific in nature, IPython is the
  interactive environment in which many scientists spend their time.
  IPython provides a powerful Python shell that integrates tightly with
  Matplotlib and with easy access to the files and operating system, and
  which can execute in a terminal or in a graphical Qt console. IPython
  also has a web-based notebook interface that can combine code with
  text, mathematical expressions, figures and multimedia. It can be
  found at \href{http://ipython.org}{ipython.org}.
\end{itemize}
While each of these tools can be installed separately, in our opinion
the most convenient way today of accessing them (especially on Windows
and Mac computers) is to install the
\href{http://www.enthought.com/products/epd\_free.php}{Free Edition of
the Enthought Python Distribution} which contain all the above. Other
free alternatives on Windows (but not on Macs) are
\href{http://code.google.com/p/pythonxy}{Python(x,y)} and
\href{http://www.lfd.uci.edu/~gohlke/pythonlibs}{Christoph Gohlke's
packages page}.

These four `core' libraries are in practice complemented by a number of
other tools for more specialized work. We will briefly list here the
ones that we think are the most commonly needed:

\begin{itemize}
\item
  Sympy: a symbolic manipulation tool that turns a Python session into a
  computer algebra system. It integrates with the IPython notebook,
  rendering results in properly typeset mathematical notation.
  \href{http://sympy.org}{sympy.org}.
\item
  Mayavi: sophisticated 3d data visualization;
  \href{http://code.enthought.com/projects/mayavi}{code.enthought.com/projects/mayavi}.
\item
  Cython: a bridge language between Python and C, useful both to
  optimize performance bottlenecks in Python and to access C libraries
  directly; \href{http://cython.org}{cython.org}.
\item
  Pandas: high-performance data structures and data analysis tools, with
  powerful data alignment and structural manipulation capabilities;
  \href{http://pandas.pydata.org}{pandas.pydata.org}.
\item
  Statsmodels: statistical data exploration and model estimation;
  \href{http://statsmodels.sourceforge.net}{statsmodels.sourceforge.net}.
\item
  Scikit-learn: general purpose machine learning algorithms with a
  common interface; \href{http://scikit-learn.org}{scikit-learn.org}.
\item
  Scikits-image: image processing toolbox;
  \href{http://scikits-image.org}{scikits-image.org}.
\item
  NetworkX: analysis of complex networks (in the graph theoretical
  sense); \href{http://networkx.lanl.gov}{networkx.lanl.gov}.
\item
  PyTables: management of hierarchical datasets using the
  industry-standard HDF5 format;
  \href{http://www.pytables.org}{www.pytables.org}.
\end{itemize}
Beyond these, for any specific problem you should look on the internet
first, before starting to write code from scratch. There's a good chance
that someone, somewhere, has written an open source library that you can
use for part or all of your problem.

\subsection{A note about the examples below}
In all subsequent examples, you will see blocks of input code, followed
by the results of the code if the code generated output. This output may
include text, graphics and other result objects. These blocks of input
can be pasted into your interactive IPython session or notebook for you
to execute. In the print version of this document, a thin vertical bar
on the left of the blocks of input and output shows which blocks go
together.

If you are reading this text as an actual IPython notebook, you can
press \texttt{Shift-Enter} or use the `play' button on the toolbar
(right-pointing triangle) to execute each block of code, known as a
`cell' in IPython:

\begin{codecell}
\begin{codeinput}
\begin{lstlisting}
# This is a block of code, below you'll see its output
print "Welcome to the world of scientific computing with Python!"
\end{lstlisting}
\end{codeinput}
\begin{codeoutput}
\begin{verbatim}
Welcome to the world of scientific computing with Python!
\end{verbatim}
\end{codeoutput}
\end{codecell}
\section{Motivation: the trapezoidal rule}
In subsequent sections we'll provide a basic introduction to the nuts
and bolts of the basic scientific python tools; but we'll first motivate
it with a brief example that illustrates what you can do in a few lines
with these tools. For this, we will use the simple problem of
approximating a definite integral with the trapezoid rule:

\[
\int_{a}^{b} f(x)\, dx \approx \frac{1}{2} \sum_{k=1}^{N} \left( x_{k} - x_{k-1} \right) \left( f(x_{k}) + f(x_{k-1}) \right).
\]

Our task will be to compute this formula for a function such as:

\[
f(x) = (x-3)(x-5)(x-7)+85
\]

integrated between $a=1$ and $b=9$.

First, we define the function and sample it evenly between 0 and 10 at
200 points:

\begin{codecell}
\begin{codeinput}
\begin{lstlisting}
def f(x):
    return (x-3)*(x-5)*(x-7)+85

import numpy as np
x = np.linspace(0, 10, 200)
y = f(x)
\end{lstlisting}
\end{codeinput}
\end{codecell}
We select $a$ and $b$, our integration limits, and we take only a few
points in that region to illustrate the error behavior of the trapezoid
approximation:

\begin{codecell}
\begin{codeinput}
\begin{lstlisting}
a, b = 1, 9
xint = x[logical_and(x>=a, x<=b)][::30]
yint = y[logical_and(x>=a, x<=b)][::30]
\end{lstlisting}
\end{codeinput}
\end{codecell}
Let's plot both the function and the area below it in the trapezoid
approximation:

\begin{codecell}
\begin{codeinput}
\begin{lstlisting}
import matplotlib.pyplot as plt
plt.plot(x, y, lw=2)
plt.axis([0, 10, 0, 140])
plt.fill_between(xint, 0, yint, facecolor='gray', alpha=0.4)
plt.text(0.5 * (a + b), 30,r"$\int_a^b f(x)dx$", horizontalalignment='center', fontsize=20);
\end{lstlisting}
\end{codeinput}
\begin{codeoutput}
\begin{center}
\includegraphics[width=0.7\textwidth]{IntroNumPy_orig_files/IntroNumPy_orig_fig_00.pdf}
\par
\end{center}
\end{codeoutput}
\end{codecell}
Compute the integral both at high accuracy and with the trapezoid
approximation

\begin{codecell}
\begin{codeinput}
\begin{lstlisting}
from scipy.integrate import quad, trapz
integral, error = quad(f, 1, 9)
trap_integral = trapz(yint, xint)
print "The integral is: %g +/- %.1e" % (integral, error)
print "The trapezoid approximation with", len(xint), "points is:", trap_integral
print "The absolute error is:", abs(integral - trap_integral)
\end{lstlisting}
\end{codeinput}
\begin{codeoutput}
\begin{verbatim}
The integral is: 680 +/- 7.5e-12
The trapezoid approximation with 6 points is: 621.286411141
The absolute error is: 58.7135888589
\end{verbatim}
\end{codeoutput}
\end{codecell}
This simple example showed us how, combining the numpy, scipy and
matplotlib libraries we can provide an illustration of a standard method
in elementary calculus with just a few lines of code. We will now
discuss with more detail the basic usage of these tools.

\section{NumPy arrays: the right data structure for scientific computing}
\subsection{Basics of Numpy arrays}
We now turn our attention to the Numpy library, which forms the base
layer for the entire `scipy ecosystem'. Once you have installed numpy,
you can import it as

\begin{codecell}
\begin{codeinput}
\begin{lstlisting}
import numpy
\end{lstlisting}
\end{codeinput}
\end{codecell}
though in this book we will use the common shorthand

\begin{codecell}
\begin{codeinput}
\begin{lstlisting}
import numpy as np
\end{lstlisting}
\end{codeinput}
\end{codecell}
As mentioned above, the main object provided by numpy is a powerful
array. We'll start by exploring how the numpy array differs from Python
lists. We start by creating a simple list and an array with the same
contents of the list:

\begin{codecell}
\begin{codeinput}
\begin{lstlisting}
lst = [10, 20, 30, 40]
arr = np.array([10, 20, 30, 40])
\end{lstlisting}
\end{codeinput}
\end{codecell}
Elements of a one-dimensional array are accessed with the same syntax as
a list:

\begin{codecell}
\begin{codeinput}
\begin{lstlisting}
lst[0]
\end{lstlisting}
\end{codeinput}
\begin{codeoutput}
\begin{verbatim}
10
\end{verbatim}
\end{codeoutput}
\end{codecell}
\begin{codecell}
\begin{codeinput}
\begin{lstlisting}
arr[0]
\end{lstlisting}
\end{codeinput}
\begin{codeoutput}
\begin{verbatim}
10
\end{verbatim}
\end{codeoutput}
\end{codecell}
\begin{codecell}
\begin{codeinput}
\begin{lstlisting}
arr[-1]
\end{lstlisting}
\end{codeinput}
\begin{codeoutput}
\begin{verbatim}
40
\end{verbatim}
\end{codeoutput}
\end{codecell}
\begin{codecell}
\begin{codeinput}
\begin{lstlisting}
arr[2:]
\end{lstlisting}
\end{codeinput}
\begin{codeoutput}
\begin{verbatim}
array([30, 40])
\end{verbatim}
\end{codeoutput}
\end{codecell}
The first difference to note between lists and arrays is that arrays are
\emph{homogeneous}; i.e.~all elements of an array must be of the same
type. In contrast, lists can contain elements of arbitrary type. For
example, we can change the last element in our list above to be a
string:

\begin{codecell}
\begin{codeinput}
\begin{lstlisting}
lst[-1] = 'a string inside a list'
lst
\end{lstlisting}
\end{codeinput}
\begin{codeoutput}
\begin{verbatim}
[10, 20, 30, 'a string inside a list']
\end{verbatim}
\end{codeoutput}
\end{codecell}
but the same can not be done with an array, as we get an error message:

\begin{codecell}
\begin{codeinput}
\begin{lstlisting}
arr[-1] = 'a string inside an array'
\end{lstlisting}
\end{codeinput}
\begin{codeoutput}
\begin{traceback}
\begin{verbatim}
---------------------------------------------------------------------------
ValueError                                Traceback (most recent call last)
/home/fperez/teach/book-math-labtool/<ipython-input-13-29c0bfa5fa8a> in <module>()
----> 1 arr[-1] = 'a string inside an array'

ValueError: invalid literal for long() with base 10: 'a string inside an array'
\end{verbatim}
\end{traceback}
\end{codeoutput}
\end{codecell}
The information about the type of an array is contained in its
\emph{dtype} attribute:

\begin{codecell}
\begin{codeinput}
\begin{lstlisting}
arr.dtype
\end{lstlisting}
\end{codeinput}
\begin{codeoutput}
\begin{verbatim}
dtype('int32')
\end{verbatim}
\end{codeoutput}
\end{codecell}
Once an array has been created, its dtype is fixed and it can only store
elements of the same type. For this example where the dtype is integer,
if we store a floating point number it will be automatically converted
into an integer:

\begin{codecell}
\begin{codeinput}
\begin{lstlisting}
arr[-1] = 1.234
arr
\end{lstlisting}
\end{codeinput}
\begin{codeoutput}
\begin{verbatim}
array([10, 20, 30,  1])
\end{verbatim}
\end{codeoutput}
\end{codecell}
Above we created an array from an existing list; now let us now see
other ways in which we can create arrays, which we'll illustrate next. A
common need is to have an array initialized with a constant value, and
very often this value is 0 or 1 (suitable as starting value for additive
and multiplicative loops respectively); \texttt{zeros} creates arrays of
all zeros, with any desired dtype:

\begin{codecell}
\begin{codeinput}
\begin{lstlisting}
np.zeros(5, float)
\end{lstlisting}
\end{codeinput}
\begin{codeoutput}
\begin{verbatim}
array([ 0.,  0.,  0.,  0.,  0.])
\end{verbatim}
\end{codeoutput}
\end{codecell}
\begin{codecell}
\begin{codeinput}
\begin{lstlisting}
np.zeros(3, int)
\end{lstlisting}
\end{codeinput}
\begin{codeoutput}
\begin{verbatim}
array([0, 0, 0])
\end{verbatim}
\end{codeoutput}
\end{codecell}
\begin{codecell}
\begin{codeinput}
\begin{lstlisting}
np.zeros(3, complex)
\end{lstlisting}
\end{codeinput}
\begin{codeoutput}
\begin{verbatim}
array([ 0.+0.j,  0.+0.j,  0.+0.j])
\end{verbatim}
\end{codeoutput}
\end{codecell}
and similarly for \texttt{ones}:

\begin{codecell}
\begin{codeinput}
\begin{lstlisting}
print '5 ones:', np.ones(5)
\end{lstlisting}
\end{codeinput}
\begin{codeoutput}
\begin{verbatim}
5 ones: [ 1.  1.  1.  1.  1.]
\end{verbatim}
\end{codeoutput}
\end{codecell}
If we want an array initialized with an arbitrary value, we can create
an empty array and then use the fill method to put the value we want
into the array:

\begin{codecell}
\begin{codeinput}
\begin{lstlisting}
a = empty(4)
a.fill(5.5)
a
\end{lstlisting}
\end{codeinput}
\begin{codeoutput}
\begin{verbatim}
array([ 5.5,  5.5,  5.5,  5.5])
\end{verbatim}
\end{codeoutput}
\end{codecell}
Numpy also offers the \texttt{arange} function, which works like the
builtin \texttt{range} but returns an array instead of a list:

\begin{codecell}
\begin{codeinput}
\begin{lstlisting}
np.arange(5)
\end{lstlisting}
\end{codeinput}
\begin{codeoutput}
\begin{verbatim}
array([0, 1, 2, 3, 4])
\end{verbatim}
\end{codeoutput}
\end{codecell}
and the \texttt{linspace} and \texttt{logspace} functions to create
linearly and logarithmically-spaced grids respectively, with a fixed
number of points and including both ends of the specified interval:

\begin{codecell}
\begin{codeinput}
\begin{lstlisting}
print "A linear grid between 0 and 1:", np.linspace(0, 1, 5)
print "A logarithmic grid between 10**1 and 10**4: ", np.logspace(1, 4, 4)
\end{lstlisting}
\end{codeinput}
\begin{codeoutput}
\begin{verbatim}
A linear grid between 0 and 1: [ 0.    0.25  0.5   0.75  1.  ]
A logarithmic grid between 10**1 and 10**4:  [    10.    100.   1000.  10000.]
\end{verbatim}
\end{codeoutput}
\end{codecell}
Finally, it is often useful to create arrays with random numbers that
follow a specific distribution. The \texttt{np.random} module contains a
number of functions that can be used to this effect, for example this
will produce an array of 5 random samples taken from a standard normal
distribution (0 mean and variance 1):

\begin{codecell}
\begin{codeinput}
\begin{lstlisting}
np.random.randn(5)
\end{lstlisting}
\end{codeinput}
\begin{codeoutput}
\begin{verbatim}
array([-0.08633343, -0.67375434,  1.00589536,  0.87081651,  1.65597822])
\end{verbatim}
\end{codeoutput}
\end{codecell}
whereas this will also give 5 samples, but from a normal distribution
with a mean of 10 and a variance of 3:

\begin{codecell}
\begin{codeinput}
\begin{lstlisting}
norm10 = np.random.normal(10, 3, 5)
norm10
\end{lstlisting}
\end{codeinput}
\begin{codeoutput}
\begin{verbatim}
array([  8.94879575,   5.53038269,   8.24847281,  12.14944165,  11.56209294])
\end{verbatim}
\end{codeoutput}
\end{codecell}
\subsection{Indexing with other arrays}
Above we saw how to index arrays with single numbers and slices, just
like Python lists. But arrays allow for a more sophisticated kind of
indexing which is very powerful: you can index an array with another
array, and in particular with an array of boolean values. This is
particluarly useful to extract information from an array that matches a
certain condition.

Consider for example that in the array \texttt{norm10} we want to
replace all values above 9 with the value 0. We can do so by first
finding the \emph{mask} that indicates where this condition is true or
false:

\begin{codecell}
\begin{codeinput}
\begin{lstlisting}
mask = norm10 > 9
mask
\end{lstlisting}
\end{codeinput}
\begin{codeoutput}
\begin{verbatim}
array([False, False, False,  True,  True], dtype=bool)
\end{verbatim}
\end{codeoutput}
\end{codecell}
Now that we have this mask, we can use it to either read those values or
to reset them to 0:

\begin{codecell}
\begin{codeinput}
\begin{lstlisting}
print 'Values above 9:', norm10[mask]
\end{lstlisting}
\end{codeinput}
\begin{codeoutput}
\begin{verbatim}
Values above 9: [ 12.14944165  11.56209294]
\end{verbatim}
\end{codeoutput}
\end{codecell}
\begin{codecell}
\begin{codeinput}
\begin{lstlisting}
print 'Resetting all values above 9 to 0...'
norm10[mask] = 0
print norm10
\end{lstlisting}
\end{codeinput}
\begin{codeoutput}
\begin{verbatim}
Resetting all values above 9 to 0...
[ 8.94879575  5.53038269  8.24847281  0.          0.        ]
\end{verbatim}
\end{codeoutput}
\end{codecell}
\subsection{Arrays with more than one dimension}
Up until now all our examples have used one-dimensional arrays. But
Numpy can create arrays of aribtrary dimensions, and all the methods
illustrated in the previous section work with more than one dimension.
For example, a list of lists can be used to initialize a two dimensional
array:

\begin{codecell}
\begin{codeinput}
\begin{lstlisting}
lst2 = [[1, 2], [3, 4]]
arr2 = np.array([[1, 2], [3, 4]])
arr2
\end{lstlisting}
\end{codeinput}
\begin{codeoutput}
\begin{verbatim}
array([[1, 2],
       [3, 4]])
\end{verbatim}
\end{codeoutput}
\end{codecell}
With two-dimensional arrays we start seeing the power of numpy: while a
nested list can be indexed using repeatedly the \texttt{{[} {]}}
operator, multidimensional arrays support a much more natural indexing
syntax with a single \texttt{{[} {]}} and a set of indices separated by
commas:

\begin{codecell}
\begin{codeinput}
\begin{lstlisting}
print lst2[0][1]
print arr2[0,1]
\end{lstlisting}
\end{codeinput}
\begin{codeoutput}
\begin{verbatim}
2
2
\end{verbatim}
\end{codeoutput}
\end{codecell}
Most of the array creation functions listed above can be used with more
than one dimension, for example:

\begin{codecell}
\begin{codeinput}
\begin{lstlisting}
np.zeros((2,3))
\end{lstlisting}
\end{codeinput}
\begin{codeoutput}
\begin{verbatim}
array([[ 0.,  0.,  0.],
       [ 0.,  0.,  0.]])
\end{verbatim}
\end{codeoutput}
\end{codecell}
\begin{codecell}
\begin{codeinput}
\begin{lstlisting}
np.random.normal(10, 3, (2, 4))
\end{lstlisting}
\end{codeinput}
\begin{codeoutput}
\begin{verbatim}
array([[ 11.26788826,   4.29619866,  11.09346496,   9.73861307],
       [ 10.54025996,   9.5146268 ,  10.80367214,  13.62204505]])
\end{verbatim}
\end{codeoutput}
\end{codecell}
In fact, the shape of an array can be changed at any time, as long as
the total number of elements is unchanged. For example, if we want a 2x4
array with numbers increasing from 0, the easiest way to create it is:

\begin{codecell}
\begin{codeinput}
\begin{lstlisting}
arr = np.arange(8).reshape(2,4)
print arr
\end{lstlisting}
\end{codeinput}
\begin{codeoutput}
\begin{verbatim}
[[0 1 2 3]
 [4 5 6 7]]
\end{verbatim}
\end{codeoutput}
\end{codecell}
With multidimensional arrays, you can also use slices, and you can mix
and match slices and single indices in the different dimensions (using
the same array as above):

\begin{codecell}
\begin{codeinput}
\begin{lstlisting}
print 'Slicing in the second row:', arr[1, 2:4]
print 'All rows, third column   :', arr[:, 2]
\end{lstlisting}
\end{codeinput}
\begin{codeoutput}
\begin{verbatim}
Slicing in the second row: [6 7]
All rows, third column   : [2 6]
\end{verbatim}
\end{codeoutput}
\end{codecell}
If you only provide one index, then you will get an array with one less
dimension containing that row:

\begin{codecell}
\begin{codeinput}
\begin{lstlisting}
print 'First row:  ', arr[0]
print 'Second row: ', arr[1]
\end{lstlisting}
\end{codeinput}
\begin{codeoutput}
\begin{verbatim}
First row:   [0 1 2 3]
Second row:  [4 5 6 7]
\end{verbatim}
\end{codeoutput}
\end{codecell}
Now that we have seen how to create arrays with more than one dimension,
it's a good idea to look at some of the most useful properties and
methods that arrays have. The following provide basic information about
the size, shape and data in the array:

\begin{codecell}
\begin{codeinput}
\begin{lstlisting}
print 'Data type                :', arr.dtype
print 'Total number of elements :', arr.size
print 'Number of dimensions     :', arr.ndim
print 'Shape (dimensionality)   :', arr.shape
print 'Memory used (in bytes)   :', arr.nbytes
\end{lstlisting}
\end{codeinput}
\begin{codeoutput}
\begin{verbatim}
Data type                : int32
Total number of elements : 8
Number of dimensions     : 2
Shape (dimensionality)   : (2, 4)
Memory used (in bytes)   : 32
\end{verbatim}
\end{codeoutput}
\end{codecell}
Arrays also have many useful methods, some especially useful ones are:

\begin{codecell}
\begin{codeinput}
\begin{lstlisting}
print 'Minimum and maximum             :', arr.min(), arr.max()
print 'Sum and product of all elements :', arr.sum(), arr.prod()
print 'Mean and standard deviation     :', arr.mean(), arr.std()
\end{lstlisting}
\end{codeinput}
\begin{codeoutput}
\begin{verbatim}
Minimum and maximum             : 0 7
Sum and product of all elements : 28 0
Mean and standard deviation     : 3.5 2.29128784748
\end{verbatim}
\end{codeoutput}
\end{codecell}
For these methods, the above operations area all computed on all the
elements of the array. But for a multidimensional array, it's possible
to do the computation along a single dimension, by passing the
\texttt{axis} parameter; for example:

\begin{codecell}
\begin{codeinput}
\begin{lstlisting}
print 'For the following array:\n', arr
print 'The sum of elements along the rows is    :', arr.sum(axis=1)
print 'The sum of elements along the columns is :', arr.sum(axis=0)
\end{lstlisting}
\end{codeinput}
\begin{codeoutput}
\begin{verbatim}
For the following array:
[[0 1 2 3]
 [4 5 6 7]]
The sum of elements along the rows is    : [ 6 22]
The sum of elements along the columns is : [ 4  6  8 10]
\end{verbatim}
\end{codeoutput}
\end{codecell}
As you can see in this example, the value of the \texttt{axis} parameter
is the dimension which will be \emph{consumed} once the operation has
been carried out. This is why to sum along the rows we use
\texttt{axis=0}.

This can be easily illustrated with an example that has more dimensions;
we create an array with 4 dimensions and shape \texttt{(3,4,5,6)} and
sum along the axis number 2 (i.e.~the \emph{third} axis, since in Python
all counts are 0-based). That consumes the dimension whose length was 5,
leaving us with a new array that has shape \texttt{(3,4,6)}:

\begin{codecell}
\begin{codeinput}
\begin{lstlisting}
np.zeros((3,4,5,6)).sum(2).shape
\end{lstlisting}
\end{codeinput}
\begin{codeoutput}
\begin{verbatim}
(3, 4, 6)
\end{verbatim}
\end{codeoutput}
\end{codecell}
Another widely used property of arrays is the \texttt{.T} attribute,
which allows you to access the transpose of the array:

\begin{codecell}
\begin{codeinput}
\begin{lstlisting}
print 'Array:\n', arr
print 'Transpose:\n', arr.T
\end{lstlisting}
\end{codeinput}
\begin{codeoutput}
\begin{verbatim}
Array:
[[0 1 2 3]
 [4 5 6 7]]
Transpose:
[[0 4]
 [1 5]
 [2 6]
 [3 7]]
\end{verbatim}
\end{codeoutput}
\end{codecell}
We don't have time here to look at all the methods and properties of
arrays, here's a complete list. Simply try exploring some of these
IPython to learn more, or read their description in the full Numpy
documentation:

\begin{verbatim}
arr.T             arr.copy          arr.getfield      arr.put           arr.squeeze
arr.all           arr.ctypes        arr.imag          arr.ravel         arr.std
arr.any           arr.cumprod       arr.item          arr.real          arr.strides
arr.argmax        arr.cumsum        arr.itemset       arr.repeat        arr.sum
arr.argmin        arr.data          arr.itemsize      arr.reshape       arr.swapaxes
arr.argsort       arr.diagonal      arr.max           arr.resize        arr.take
arr.astype        arr.dot           arr.mean          arr.round         arr.tofile
arr.base          arr.dtype         arr.min           arr.searchsorted  arr.tolist
arr.byteswap      arr.dump          arr.nbytes        arr.setasflat     arr.tostring
arr.choose        arr.dumps         arr.ndim          arr.setfield      arr.trace
arr.clip          arr.fill          arr.newbyteorder  arr.setflags      arr.transpose
arr.compress      arr.flags         arr.nonzero       arr.shape         arr.var
arr.conj          arr.flat          arr.prod          arr.size          arr.view
arr.conjugate     arr.flatten       arr.ptp           arr.sort          
\end{verbatim}


\subsection{Operating with arrays}
Arrays support all regular arithmetic operators, and the numpy library
also contains a complete collection of basic mathematical functions that
operate on arrays. It is important to remember that in general, all
operations with arrays are applied \emph{element-wise}, i.e., are
applied to all the elements of the array at the same time. Consider for
example:

\begin{codecell}
\begin{codeinput}
\begin{lstlisting}
arr1 = np.arange(4)
arr2 = np.arange(10, 14)
print arr1, '+', arr2, '=', arr1+arr2
\end{lstlisting}
\end{codeinput}
\begin{codeoutput}
\begin{verbatim}
[0 1 2 3] + [10 11 12 13] = [10 12 14 16]
\end{verbatim}
\end{codeoutput}
\end{codecell}
Importantly, you must remember that even the multiplication operator is
by default applied element-wise, it is \emph{not} the matrix
multiplication from linear algebra (as is the case in Matlab, for
example):

\begin{codecell}
\begin{codeinput}
\begin{lstlisting}
print arr1, '*', arr2, '=', arr1*arr2
\end{lstlisting}
\end{codeinput}
\begin{codeoutput}
\begin{verbatim}
[0 1 2 3] * [10 11 12 13] = [ 0 11 24 39]
\end{verbatim}
\end{codeoutput}
\end{codecell}
While this means that in principle arrays must always match in their
dimensionality in order for an operation to be valid, numpy will
\emph{broadcast} dimensions when possible. For example, suppose that you
want to add the number 1.5 to \texttt{arr1}; the following would be a
valid way to do it:

\begin{codecell}
\begin{codeinput}
\begin{lstlisting}
arr1 + 1.5*np.ones(4)
\end{lstlisting}
\end{codeinput}
\begin{codeoutput}
\begin{verbatim}
array([ 1.5,  2.5,  3.5,  4.5])
\end{verbatim}
\end{codeoutput}
\end{codecell}
But thanks to numpy's broadcasting rules, the following is equally
valid:

\begin{codecell}
\begin{codeinput}
\begin{lstlisting}
arr1 + 1.5
\end{lstlisting}
\end{codeinput}
\begin{codeoutput}
\begin{verbatim}
array([ 1.5,  2.5,  3.5,  4.5])
\end{verbatim}
\end{codeoutput}
\end{codecell}
In this case, numpy looked at both operands and saw that the first
(\texttt{arr1}) was a one-dimensional array of length 4 and the second
was a scalar, considered a zero-dimensional object. The broadcasting
rules allow numpy to:

\begin{itemize}
\item
  \emph{create} new dimensions of length 1 (since this doesn't change
  the size of the array)
\item
  `stretch' a dimension of length 1 that needs to be matched to a
  dimension of a different size.
\end{itemize}
So in the above example, the scalar 1.5 is effectively:

\begin{itemize}
\item
  first `promoted' to a 1-dimensional array of length 1
\item
  then, this array is `stretched' to length 4 to match the dimension of
  \texttt{arr1}.
\end{itemize}
After these two operations are complete, the addition can proceed as now
both operands are one-dimensional arrays of length 4.

This broadcasting behavior is in practice enormously powerful,
especially because when numpy broadcasts to create new dimensions or to
`stretch' existing ones, it doesn't actually replicate the data. In the
example above the operation is carried \emph{as if} the 1.5 was a 1-d
array with 1.5 in all of its entries, but no actual array was ever
created. This can save lots of memory in cases when the arrays in
question are large and can have significant performance implications.

The general rule is: when operating on two arrays, NumPy compares their
shapes element-wise. It starts with the trailing dimensions, and works
its way forward, creating dimensions of length 1 as needed. Two
dimensions are considered compatible when

\begin{itemize}
\item
  they are equal to begin with, or
\item
  one of them is 1; in this case numpy will do the `stretching' to make
  them equal.
\end{itemize}
If these conditions are not met, a
\texttt{ValueError: frames are not aligned} exception is thrown,
indicating that the arrays have incompatible shapes. The size of the
resulting array is the maximum size along each dimension of the input
arrays.

This shows how the broadcasting rules work in several dimensions:

\begin{codecell}
\begin{codeinput}
\begin{lstlisting}
b = np.array([2, 3, 4, 5])
print arr, '\n\n+', b , '\n----------------\n', arr + b
\end{lstlisting}
\end{codeinput}
\begin{codeoutput}
\begin{verbatim}
[[0 1 2 3]
 [4 5 6 7]] 

+ [2 3 4 5] 
----------------
[[ 2  4  6  8]
 [ 6  8 10 12]]
\end{verbatim}
\end{codeoutput}
\end{codecell}
Now, how could you use broadcasting to say add \texttt{{[}4, 6{]}} along
the rows to \texttt{arr} above? Simply performing the direct addition
will produce the error we previously mentioned:

\begin{codecell}
\begin{codeinput}
\begin{lstlisting}
c = np.array([4, 6])
arr + c
\end{lstlisting}
\end{codeinput}
\begin{codeoutput}
\begin{traceback}
\begin{verbatim}
---------------------------------------------------------------------------
ValueError                                Traceback (most recent call last)
/home/fperez/teach/book-math-labtool/<ipython-input-45-62aa20ac1980> in <module>()
      1 c = np.array([4, 6])
----> 2 arr + c

ValueError: operands could not be broadcast together with shapes (2,4) (2) 
\end{verbatim}
\end{traceback}
\end{codeoutput}
\end{codecell}
According to the rules above, the array \texttt{c} would need to have a
\emph{trailing} dimension of 1 for the broadcasting to work. It turns
out that numpy allows you to `inject' new dimensions anywhere into an
array on the fly, by indexing it with the special object
\texttt{np.newaxis}:

\begin{codecell}
\begin{codeinput}
\begin{lstlisting}
(c[:, np.newaxis]).shape
\end{lstlisting}
\end{codeinput}
\begin{codeoutput}
\begin{verbatim}
(2, 1)
\end{verbatim}
\end{codeoutput}
\end{codecell}
This is exactly what we need, and indeed it works:

\begin{codecell}
\begin{codeinput}
\begin{lstlisting}
arr + c[:, np.newaxis]
\end{lstlisting}
\end{codeinput}
\begin{codeoutput}
\begin{verbatim}
array([[ 4,  5,  6,  7],
       [10, 11, 12, 13]])
\end{verbatim}
\end{codeoutput}
\end{codecell}
For the full broadcasting rules, please see the official Numpy docs,
which describe them in detail and with more complex examples.

As we mentioned before, Numpy ships with a full complement of
mathematical functions that work on entire arrays, including logarithms,
exponentials, trigonometric and hyperbolic trigonometric functions, etc.
Furthermore, scipy ships a rich special function library in the
\texttt{scipy.special} module that includes Bessel, Airy, Fresnel,
Laguerre and other classical special functions. For example, sampling
the sine function at 100 points between $0$ and $2\pi$ is as simple as:

\begin{codecell}
\begin{codeinput}
\begin{lstlisting}
x = np.linspace(0, 2*np.pi, 100)
y = np.sin(x)
\end{lstlisting}
\end{codeinput}
\end{codecell}
\subsection{Linear algebra in numpy}
Numpy ships with a basic linear algebra library, and all arrays have a
\texttt{dot} method whose behavior is that of the scalar dot product
when its arguments are vectors (one-dimensional arrays) and the
traditional matrix multiplication when one or both of its arguments are
two-dimensional arrays:

\begin{codecell}
\begin{codeinput}
\begin{lstlisting}
v1 = np.array([2, 3, 4])
v2 = np.array([1, 0, 1])
print v1, '.', v2, '=', v1.dot(v2)
\end{lstlisting}
\end{codeinput}
\begin{codeoutput}
\begin{verbatim}
[2 3 4] . [1 0 1] = 6
\end{verbatim}
\end{codeoutput}
\end{codecell}
Here is a regular matrix-vector multiplication, note that the array
\texttt{v1} should be viewed as a \emph{column} vector in traditional
linear algebra notation; numpy makes no distinction between row and
column vectors and simply verifies that the dimensions match the
required rules of matrix multiplication, in this case we have a
$2 \times 3$ matrix multiplied by a 3-vector, which produces a 2-vector:

\begin{codecell}
\begin{codeinput}
\begin{lstlisting}
A = np.arange(6).reshape(2, 3)
print A, 'x', v1, '=', A.dot(v1)
\end{lstlisting}
\end{codeinput}
\begin{codeoutput}
\begin{verbatim}
[[0 1 2]
 [3 4 5]] x [2 3 4] = [11 38]
\end{verbatim}
\end{codeoutput}
\end{codecell}
For matrix-matrix multiplication, the same dimension-matching rules must
be satisfied, e.g.~consider the difference between $A \times A^T$:

\begin{codecell}
\begin{codeinput}
\begin{lstlisting}
print A.dot(A.T)
\end{lstlisting}
\end{codeinput}
\begin{codeoutput}
\begin{verbatim}
[[ 5 14]
 [14 50]]
\end{verbatim}
\end{codeoutput}
\end{codecell}
and $A^T \times A$:

\begin{codecell}
\begin{codeinput}
\begin{lstlisting}
print A.T.dot(A)
\end{lstlisting}
\end{codeinput}
\begin{codeoutput}
\begin{verbatim}
[[ 9 12 15]
 [12 17 22]
 [15 22 29]]
\end{verbatim}
\end{codeoutput}
\end{codecell}
Furthermore, the \texttt{numpy.linalg} module includes additional
functionality such as determinants, matrix norms, Cholesky, eigenvalue
and singular value decompositions, etc. For even more linear algebra
tools, \texttt{scipy.linalg} contains the majority of the tools in the
classic LAPACK libraries as well as functions to operate on sparse
matrices. We refer the reader to the Numpy and Scipy documentations for
additional details on these.

\subsection{Reading and writing arrays to disk}
Numpy lets you read and write arrays into files in a number of ways. In
order to use these tools well, it is critical to understand the
difference between a \emph{text} and a \emph{binary} file containing
numerical data. In a text file, the number $\pi$ could be written as
``3.141592653589793'', for example: a string of digits that a human can
read, with in this case 15 decimal digits. In contrast, that same number
written to a binary file would be encoded as 8 characters (bytes) that
are not readable by a human but which contain the exact same data that
the variable \texttt{pi} had in the computer's memory.

The tradeoffs between the two modes are thus:

\begin{itemize}
\item
  Text mode: occupies more space, precision can be lost (if not all
  digits are written to disk), but is readable and editable by hand with
  a text editor. Can \emph{only} be used for one- and two-dimensional
  arrays.
\item
  Binary mode: compact and exact representation of the data in memory,
  can't be read or edited by hand. Arrays of any size and dimensionality
  can be saved and read without loss of information.
\end{itemize}
First, let's see how to read and write arrays in text mode. The
\texttt{np.savetxt} function saves an array to a text file, with options
to control the precision, separators and even adding a header:

\begin{codecell}
\begin{codeinput}
\begin{lstlisting}
arr = np.arange(10).reshape(2, 5)
np.savetxt('test.out', arr, fmt='%.2e', header="My dataset")
!cat test.out
\end{lstlisting}
\end{codeinput}
\begin{codeoutput}
\begin{verbatim}
# My dataset
0.00e+00 1.00e+00 2.00e+00 3.00e+00 4.00e+00
5.00e+00 6.00e+00 7.00e+00 8.00e+00 9.00e+00
\end{verbatim}
\end{codeoutput}
\end{codecell}
And this same type of file can then be read with the matching
\texttt{np.loadtxt} function:

\begin{codecell}
\begin{codeinput}
\begin{lstlisting}
arr2 = np.loadtxt('test.out')
print arr2
\end{lstlisting}
\end{codeinput}
\begin{codeoutput}
\begin{verbatim}
[[ 0.  1.  2.  3.  4.]
 [ 5.  6.  7.  8.  9.]]
\end{verbatim}
\end{codeoutput}
\end{codecell}
For binary data, Numpy provides the \texttt{np.save} and
\texttt{np.savez} routines. The first saves a single array to a file
with \texttt{.npy} extension, while the latter can be used to save a
\emph{group} of arrays into a single file with \texttt{.npz} extension.
The files created with these routines can then be read with the
\texttt{np.load} function.

Let us first see how to use the simpler \texttt{np.save} function to
save a single array:

\begin{codecell}
\begin{codeinput}
\begin{lstlisting}
np.save('test.npy', arr2)
# Now we read this back
arr2n = np.load('test.npy')
# Let's see if any element is non-zero in the difference.
# A value of True would be a problem.
print 'Any differences?', np.any(arr2-arr2n)
\end{lstlisting}
\end{codeinput}
\begin{codeoutput}
\begin{verbatim}
Any differences? False
\end{verbatim}
\end{codeoutput}
\end{codecell}
Now let us see how the \texttt{np.savez} function works. You give it a
filename and either a sequence of arrays or a set of keywords. In the
first mode, the function will auotmatically name the saved arrays in the
archive as \texttt{arr\_0}, \texttt{arr\_1}, etc:

\begin{codecell}
\begin{codeinput}
\begin{lstlisting}
np.savez('test.npz', arr, arr2)
arrays = np.load('test.npz')
arrays.files
\end{lstlisting}
\end{codeinput}
\begin{codeoutput}
\begin{verbatim}
['arr_1', 'arr_0']
\end{verbatim}
\end{codeoutput}
\end{codecell}
Alternatively, we can explicitly choose how to name the arrays we save:

\begin{codecell}
\begin{codeinput}
\begin{lstlisting}
np.savez('test.npz', array1=arr, array2=arr2)
arrays = np.load('test.npz')
arrays.files
\end{lstlisting}
\end{codeinput}
\begin{codeoutput}
\begin{verbatim}
['array2', 'array1']
\end{verbatim}
\end{codeoutput}
\end{codecell}
The object returned by \texttt{np.load} from an \texttt{.npz} file works
like a dictionary, though you can also access its constituent files by
attribute using its special \texttt{.f} field; this is best illustrated
with an example with the \texttt{arrays} object from above:

\begin{codecell}
\begin{codeinput}
\begin{lstlisting}
print 'First row of first array:', arrays['array1'][0]
# This is an equivalent way to get the same field
print 'First row of first array:', arrays.f.array1[0]
\end{lstlisting}
\end{codeinput}
\begin{codeoutput}
\begin{verbatim}
First row of first array: [0 1 2 3 4]
First row of first array: [0 1 2 3 4]
\end{verbatim}
\end{codeoutput}
\end{codecell}
This \texttt{.npz} format is a very convenient way to package compactly
and without loss of information, into a single file, a group of related
arrays that pertain to a specific problem. At some point, however, the
complexity of your dataset may be such that the optimal approach is to
use one of the standard formats in scientific data processing that have
been designed to handle complex datasets, such as NetCDF or HDF5.

Fortunately, there are tools for manipulating these formats in Python,
and for storing data in other ways such as databases. A complete
discussion of the possibilities is beyond the scope of this discussion,
but of particular interest for scientific users we at least mention the
following:

\begin{itemize}
\item
  The \texttt{scipy.io} module contains routines to read and write
  Matlab files in \texttt{.mat} format and files in the NetCDF format
  that is widely used in certain scientific disciplines.
\item
  For manipulating files in the HDF5 format, there are two excellent
  options in Python: The PyTables project offers a high-level, object
  oriented approach to manipulating HDF5 datasets, while the h5py
  project offers a more direct mapping to the standard HDF5 library
  interface. Both are excellent tools; if you need to work with HDF5
  datasets you should read some of their documentation and examples and
  decide which approach is a better match for your needs.
\end{itemize}

\section{High quality data visualization with Matplotlib}
The \href{http://matplotlib.sf.net}{matplotlib} library is a powerful
tool capable of producing complex publication-quality figures with fine
layout control in two and three dimensions; here we will only provide a
minimal self-contained introduction to its usage that covers the
functionality needed for the rest of the book. We encourage the reader
to read the tutorials included with the matplotlib documentation as well
as to browse its extensive gallery of examples that include source code.

Just as we typically use the shorthand \texttt{np} for Numpy, we will
use \texttt{plt} for the \texttt{matplotlib.pyplot} module where the
easy-to-use plotting functions reside (the library contains a rich
object-oriented architecture that we don't have the space to discuss
here):

\begin{codecell}
\begin{codeinput}
\begin{lstlisting}
import matplotlib.pyplot as plt
\end{lstlisting}
\end{codeinput}
\end{codecell}
The most frequently used function is simply called \texttt{plot}, here
is how you can make a simple plot of $\sin(x)$ for $x \in [0, 2\pi]$
with labels and a grid (we use the semicolon in the last line to
suppress the display of some information that is unnecessary right now):

\begin{codecell}
\begin{codeinput}
\begin{lstlisting}
x = np.linspace(0, 2*np.pi)
y = np.sin(x)
plt.plot(x,y, label='sin(x)')
plt.legend()
plt.grid()
plt.title('Harmonic')
plt.xlabel('x')
plt.ylabel('y');
\end{lstlisting}
\end{codeinput}
\begin{codeoutput}
\begin{center}
\includegraphics[width=0.7\textwidth]{IntroNumPy_orig_files/IntroNumPy_orig_fig_01.pdf}
\par
\end{center}
\end{codeoutput}
\end{codecell}
You can control the style, color and other properties of the markers,
for example:

\begin{codecell}
\begin{codeinput}
\begin{lstlisting}
plt.plot(x, y, linewidth=2);
\end{lstlisting}
\end{codeinput}
\begin{codeoutput}
\begin{center}
\includegraphics[width=0.7\textwidth]{IntroNumPy_orig_files/IntroNumPy_orig_fig_02.pdf}
\par
\end{center}
\end{codeoutput}
\end{codecell}
\begin{codecell}
\begin{codeinput}
\begin{lstlisting}
plt.plot(x, y, 'o', markersize=5, color='r');
\end{lstlisting}
\end{codeinput}
\begin{codeoutput}
\begin{center}
\includegraphics[width=0.7\textwidth]{IntroNumPy_orig_files/IntroNumPy_orig_fig_03.pdf}
\par
\end{center}
\end{codeoutput}
\end{codecell}
We will now see how to create a few other common plot types, such as a
simple error plot:

\begin{codecell}
\begin{codeinput}
\begin{lstlisting}
# example data
x = np.arange(0.1, 4, 0.5)
y = np.exp(-x)

# example variable error bar values
yerr = 0.1 + 0.2*np.sqrt(x)
xerr = 0.1 + yerr

# First illustrate basic pyplot interface, using defaults where possible.
plt.figure()
plt.errorbar(x, y, xerr=0.2, yerr=0.4)
plt.title("Simplest errorbars, 0.2 in x, 0.4 in y");
\end{lstlisting}
\end{codeinput}
\begin{codeoutput}
\begin{center}
\includegraphics[width=0.7\textwidth]{IntroNumPy_orig_files/IntroNumPy_orig_fig_04.pdf}
\par
\end{center}
\end{codeoutput}
\end{codecell}
A simple log plot

\begin{codecell}
\begin{codeinput}
\begin{lstlisting}
x = np.linspace(-5, 5)
y = np.exp(-x**2)
plt.semilogy(x, y);
\end{lstlisting}
\end{codeinput}
\begin{codeoutput}
\begin{center}
\includegraphics[width=0.7\textwidth]{IntroNumPy_orig_files/IntroNumPy_orig_fig_05.pdf}
\par
\end{center}
\end{codeoutput}
\end{codecell}
A histogram annotated with text inside the plot, using the \texttt{text}
function:

\begin{codecell}
\begin{codeinput}
\begin{lstlisting}
mu, sigma = 100, 15
x = mu + sigma * np.random.randn(10000)

# the histogram of the data
n, bins, patches = plt.hist(x, 50, normed=1, facecolor='g', alpha=0.75)

plt.xlabel('Smarts')
plt.ylabel('Probability')
plt.title('Histogram of IQ')
# This will put a text fragment at the position given:
plt.text(55, .027, r'$\mu=100,\ \sigma=15$', fontsize=14)
plt.axis([40, 160, 0, 0.03])
plt.grid(True)
\end{lstlisting}
\end{codeinput}
\begin{codeoutput}
\begin{center}
\includegraphics[width=0.7\textwidth]{IntroNumPy_orig_files/IntroNumPy_orig_fig_06.pdf}
\par
\end{center}
\end{codeoutput}
\end{codecell}
\subsection{Image display}
The \texttt{imshow} command can display single or multi-channel images.
A simple array of random numbers, plotted in grayscale:

\begin{codecell}
\begin{codeinput}
\begin{lstlisting}
from matplotlib import cm
plt.imshow(np.random.rand(5, 10), cmap=cm.gray, interpolation='nearest');
\end{lstlisting}
\end{codeinput}
\begin{codeoutput}
\begin{center}
\includegraphics[width=0.7\textwidth]{IntroNumPy_orig_files/IntroNumPy_orig_fig_07.pdf}
\par
\end{center}
\end{codeoutput}
\end{codecell}
A real photograph is a multichannel image, \texttt{imshow} interprets it
correctly:

\begin{codecell}
\begin{codeinput}
\begin{lstlisting}
img = plt.imread('stinkbug.png')
print 'Dimensions of the array img:', img.shape
plt.imshow(img);
\end{lstlisting}
\end{codeinput}
\begin{codeoutput}
\begin{verbatim}
Dimensions of the array img: (375, 500, 3)
\end{verbatim}
\begin{center}
\includegraphics[width=0.7\textwidth]{IntroNumPy_orig_files/IntroNumPy_orig_fig_08.pdf}
\par
\end{center}
\end{codeoutput}
\end{codecell}
\subsection{Simple 3d plotting with matplotlib}
Note that you must execute at least once in your session:

\begin{codecell}
\begin{codeinput}
\begin{lstlisting}
from mpl_toolkits.mplot3d import Axes3D
\end{lstlisting}
\end{codeinput}
\end{codecell}
One this has been done, you can create 3d axes with the
\texttt{projection='3d'} keyword to \texttt{add\_subplot}:

\begin{verbatim}
fig = plt.figure()
fig.add_subplot(<other arguments here>, projection='3d')
\end{verbatim}


A simple surface plot:

\begin{codecell}
\begin{codeinput}
\begin{lstlisting}
from mpl_toolkits.mplot3d.axes3d import Axes3D
from matplotlib import cm

fig = plt.figure()
ax = fig.add_subplot(1, 1, 1, projection='3d')
X = np.arange(-5, 5, 0.25)
Y = np.arange(-5, 5, 0.25)
X, Y = np.meshgrid(X, Y)
R = np.sqrt(X**2 + Y**2)
Z = np.sin(R)
surf = ax.plot_surface(X, Y, Z, rstride=1, cstride=1, cmap=cm.jet,
        linewidth=0, antialiased=False)
ax.set_zlim3d(-1.01, 1.01);
\end{lstlisting}
\end{codeinput}
\begin{codeoutput}
\begin{center}
\includegraphics[width=0.7\textwidth]{IntroNumPy_orig_files/IntroNumPy_orig_fig_09.pdf}
\par
\end{center}
\end{codeoutput}
\end{codecell}
\section{IPython: a powerful interactive environment}
A key component of the everyday workflow of most scientific computing
environments is a good interactive environment, that is, a system in
which you can execute small amounts of code and view the results
immediately, combining both printing out data and opening graphical
visualizations. All modern systems for scientific computing, commercial
and open source, include such functionality.

Out of the box, Python also offers a simple interactive shell with very
limited capabilities. But just like the scientific community built Numpy
to provide arrays suited for scientific work (since Pytyhon's lists
aren't optimal for this task), it has also developed an interactive
environment much more sophisticated than the built-in one. The
\href{http://ipython.org}{IPython project} offers a set of tools to make
productive use of the Python language, all the while working
interactively and with immedate feedback on your results. The basic
tools that IPython provides are:

\begin{enumerate}[1.]
\item
  A powerful terminal shell, with many features designed to increase the
  fluidity and productivity of everyday scientific workflows, including:

  \begin{itemize}
  \item
    rich introspection of all objects and variables including easy
    access to the source code of any function
  \item
    powerful and extensible tab completion of variables and filenames,
  \item
    tight integration with matplotlib, supporting interactive figures
    that don't block the terminal,
  \item
    direct access to the filesystem and underlying operating system,
  \item
    an extensible system for shell-like commands called `magics' that
    reduce the work needed to perform many common tasks,
  \item
    tools for easily running, timing, profiling and debugging your
    codes,
  \item
    syntax highlighted error messages with much more detail than the
    default Python ones,
  \item
    logging and access to all previous history of inputs, including
    across sessions
  \end{itemize}
\item
  A Qt console that provides the look and feel of a terminal, but adds
  support for inline figures, graphical calltips, a persistent session
  that can survive crashes (even segfaults) of the kernel process, and
  more.
\item
  A web-based notebook that can execute code and also contain rich text
  and figures, mathematical equations and arbitrary HTML. This notebook
  presents a document-like view with cells where code is executed but
  that can be edited in-place, reordered, mixed with explanatory text
  and figures, etc.
\item
  A high-performance, low-latency system for parallel computing that
  supports the control of a cluster of IPython engines communicating
  over a network, with optimizations that minimize unnecessary copying
  of large objects (especially numpy arrays).
\end{enumerate}
We will now discuss the highlights of the tools 1-3 above so that you
can make them an effective part of your workflow. The topic of parallel
computing is beyond the scope of this document, but we encourage you to
read the extensive
\href{http://ipython.org/ipython-doc/rel-0.12.1/parallel/index.html}{documentation}
and \href{http://minrk.github.com/scipy-tutorial-2011/}{tutorials} on
this available on the IPython website.

\subsection{The IPython terminal}
You can start IPython at the terminal simply by typing:

\begin{verbatim}
$ ipython
\end{verbatim}
which will provide you some basic information about how to get started
and will then open a prompt labeled \texttt{In {[}1{]}:} for you to
start typing. Here we type $2^{64}$ and Python computes the result for
us in exact arithmetic, returning it as \texttt{Out{[}1{]}}:

\begin{verbatim}
$ ipython
Python 2.7.2+ (default, Oct  4 2011, 20:03:08) 
Type "copyright", "credits" or "license" for more information.

IPython 0.13.dev -- An enhanced Interactive Python.
?         -> Introduction and overview of IPython's features.
%quickref -> Quick reference.
help      -> Python's own help system.
object?   -> Details about 'object', use 'object??' for extra details.

In [1]: 2**64
Out[1]: 18446744073709551616L
\end{verbatim}
The first thing you should know about IPython is that all your inputs
and outputs are saved. There are two variables named \texttt{In} and
\texttt{Out} which are filled as you work with your results.
Furthermore, all outputs are also saved to auto-created variables of the
form \texttt{\_NN} where \texttt{NN} is the prompt number, and inputs to
\texttt{\_iNN}. This allows you to recover quickly the result of a prior
computation by referring to its number even if you forgot to store it as
a variable. For example, later on in the above session you can do:

\begin{verbatim}
In [6]: print _1
18446744073709551616
\end{verbatim}


We strongly recommend that you take a few minutes to read at least the
basic introduction provided by the \texttt{?} command, and keep in mind
that the \texttt{\%quickref} command at all times can be used as a quick
reference ``cheat sheet'' of the most frequently used features of
IPython.

At the IPython prompt, any valid Python code that you type will be
executed similarly to the default Python shell (though often with more
informative feedback). But since IPython is a \emph{superset} of the
default Python shell; let's have a brief look at some of its additional
functionality.

\textbf{Object introspection}

A simple \texttt{?} command provides a general introduction to IPython,
but as indicated in the banner above, you can use the \texttt{?} syntax
to ask for details about any object. For example, if we type
\texttt{\_1?}, IPython will print the following details about this
variable:

\begin{verbatim}
In [14]: _1?
Type:       long
Base Class: <type 'long'>
String Form:18446744073709551616
Namespace:  Interactive
Docstring:
long(x[, base]) -> integer

Convert a string or number to a long integer, if possible.  A floating

[etc... snipped for brevity]
\end{verbatim}
If you add a second \texttt{?} and for any oobject \texttt{x} type
\texttt{x??}, IPython will try to provide an even more detailed analsysi
of the object, including its syntax-highlighted source code when it can
be found. It's possible that \texttt{x??} returns the same information
as \texttt{x?}, but in many cases \texttt{x??} will indeed provide
additional details.

Finally, the \texttt{?} syntax is also useful to search
\emph{namespaces} with wildcards. Suppose you are wondering if there is
any function in Numpy that may do text-related things; with
\texttt{np.*txt*?}, IPython will print all the names in the \texttt{np}
namespace (our Numpy shorthand) that have `txt' anywhere in their name:

\begin{verbatim}
In [17]: np.*txt*?
np.genfromtxt
np.loadtxt
np.mafromtxt
np.ndfromtxt
np.recfromtxt
np.savetxt
\end{verbatim}


\textbf{Tab completion}

IPython makes the tab key work extra hard for you as a way to rapidly
inspect objects and libraries. Whenever you have typed something at the
prompt, by hitting the \texttt{\textless{}tab\textgreater{}} key IPython
will try to complete the rest of the line. For this, IPython will
analyze the text you had so far and try to search for Python data or
files that may match the context you have already provided.

For example, if you type \texttt{np.load} and hit the key, you'll see:

\begin{verbatim}
In [21]: np.load<TAB HERE>
np.load     np.loads    np.loadtxt  
\end{verbatim}
so you can quickly find all the load-related functionality in numpy. Tab
completion works even for function arguments, for example consider this
function definition:

\begin{verbatim}
In [20]: def f(x, frobinate=False):
   ....:     if frobinate:
   ....:         return x**2
   ....:     
\end{verbatim}
If you now use the \texttt{\textless{}tab\textgreater{}} key after
having typed `fro' you'll get all valid Python completions, but those
marked with \texttt{=} at the end are known to be keywords of your
function:

\begin{verbatim}
In [21]: f(2, fro<TAB HERE>
frobinate=    frombuffer    fromfunction  frompyfunc    fromstring    
from          fromfile      fromiter      fromregex     frozenset     
\end{verbatim}
at this point you can add the \texttt{b} letter and hit
\texttt{\textless{}tab\textgreater{}} once more, and IPython will finish
the line for you:

\begin{verbatim}
In [21]: f(2, frobinate=
\end{verbatim}
As a beginner, simply get into the habit of using
\texttt{\textless{}tab\textgreater{}} after most objects; it should
quickly become second nature as you will see how helps keep a fluid
workflow and discover useful information. Later on you can also
customize this behavior by writing your own completion code, if you so
desire.

\textbf{Matplotlib integration}

One of the most useful features of IPython for scientists is its tight
integration with matplotlib: at the terminal IPython lets you open
matplotlib figures without blocking your typing (which is what happens
if you try to do the same thing at the default Python shell), and in the
Qt console and notebook you can even view your figures embedded in your
workspace next to the code that created them.

The matplotlib support can be either activated when you start IPython by
passing the \texttt{-{}-pylab} flag, or at any point later in your
session by using the \texttt{\%pylab} command. If you start IPython with
\texttt{-{}-pylab}, you'll see something like this (note the extra
message about pylab):

\begin{verbatim}
$ ipython --pylab
Python 2.7.2+ (default, Oct  4 2011, 20:03:08) 
Type "copyright", "credits" or "license" for more information.

IPython 0.13.dev -- An enhanced Interactive Python.
?         -> Introduction and overview of IPython's features.
%quickref -> Quick reference.
help      -> Python's own help system.
object?   -> Details about 'object', use 'object??' for extra details.

Welcome to pylab, a matplotlib-based Python environment [backend: Qt4Agg].
For more information, type 'help(pylab)'.

In [1]: 
\end{verbatim}
Furthermore, IPython will import \texttt{numpy} with the \texttt{np}
shorthand, \texttt{matplotlib.pyplot} as \texttt{plt}, and it will also
load all of the numpy and pyplot top-level names so that you can
directly type something like:

\begin{verbatim}
In [1]: x = linspace(0, 2*pi, 200)

In [2]: plot(x, sin(x))
Out[2]: [<matplotlib.lines.Line2D at 0x9e7c16c>]
\end{verbatim}
instead of having to prefix each call with its full signature (as we
have been doing in the examples thus far):

\begin{verbatim}
In [3]: x = np.linspace(0, 2*np.pi, 200)

In [4]: plt.plot(x, np.sin(x))
Out[4]: [<matplotlib.lines.Line2D at 0x9e900ac>]
\end{verbatim}
This shorthand notation can be a huge time-saver when working
interactively (it's a few characters but you are likely to type them
hundreds of times in a session). But we should note that as you develop
persistent scripts and notebooks meant for reuse, it's best to get in
the habit of using the longer notation (known as \emph{fully qualified
names} as it's clearer where things come from and it makes for more
robust, readable and maintainable code in the long run).

\textbf{Access to the operating system and files}

In IPython, you can type \texttt{ls} to see your files or \texttt{cd} to
change directories, just like you would at a regular system prompt:

\begin{verbatim}
In [2]: cd tests
/home/fperez/ipython/nbconvert/tests

In [3]: ls test.*
test.aux  test.html  test.ipynb  test.log  test.out  test.pdf  test.rst  test.tex
\end{verbatim}
Furthermore, if you use the \texttt{!} at the beginning of a line, any
commands you pass afterwards go directly to the operating system:

\begin{verbatim}
In [4]: !echo "Hello IPython"
Hello IPython
\end{verbatim}
IPython offers a useful twist in this feature: it will substitute in the
command the value of any \emph{Python} variable you may have if you
prepend it with a \texttt{\$} sign:

\begin{verbatim}
In [5]: message = 'IPython interpolates from Python to the shell'

In [6]: !echo $message
IPython interpolates from Python to the shell
\end{verbatim}
This feature can be extremely useful, as it lets you combine the power
and clarity of Python for complex logic with the immediacy and
familiarity of many shell commands. Additionally, if you start the line
with \emph{two} \texttt{\$\$} signs, the output of the command will be
automatically captured as a list of lines, e.g.:

\begin{verbatim}
In [10]: !!ls test.*
Out[10]: 
['test.aux',
 'test.html',
 'test.ipynb',
 'test.log',
 'test.out',
 'test.pdf',
 'test.rst',
 'test.tex']
\end{verbatim}
As explained above, you can now use this as the variable \texttt{\_10}.
If you directly want to capture the output of a system command to a
Python variable, you can use the syntax \texttt{=!}:

\begin{verbatim}
In [11]: testfiles =! ls test.*

In [12]: print testfiles
['test.aux', 'test.html', 'test.ipynb', 'test.log', 'test.out', 'test.pdf', 'test.rst', 'test.tex']
\end{verbatim}
Finally, the special \texttt{\%alias} command lets you define names that
are shorthands for system commands, so that you can type them without
having to prefix them via \texttt{!} explicitly (for example,
\texttt{ls} is an alias that has been predefined for you at startup).

\textbf{Magic commands}

IPython has a system for special commands, called `magics', that let you
control IPython itself and perform many common tasks with a more
shell-like syntax: it uses spaces for delimiting arguments, flags can be
set with dashes and all arguments are treated as strings, so no
additional quoting is required. This kind of syntax is invalid in the
Python language but very convenient for interactive typing (less
parentheses, commans and quoting everywhere); IPython distinguishes the
two by detecting lines that start with the \texttt{\%} character.

You can learn more about the magic system by simply typing
\texttt{\%magic} at the prompt, which will give you a short description
plus the documentation on \emph{all} available magics. If you want to
see only a listing of existing magics, you can use \texttt{\%lsmagic}:

\begin{verbatim}
In [4]: lsmagic
Available magic functions:
%alias  %autocall  %autoindent  %automagic  %bookmark  %c  %cd  %colors  %config  %cpaste
%debug  %dhist  %dirs  %doctest_mode  %ds  %ed  %edit  %env  %gui  %hist  %history
%install_default_config  %install_ext  %install_profiles  %load_ext  %loadpy  %logoff  %logon  
%logstart  %logstate  %logstop  %lsmagic  %macro  %magic  %notebook  %page  %paste  %pastebin  
%pd  %pdb  %pdef  %pdoc  %pfile  %pinfo  %pinfo2  %pop  %popd  %pprint  %precision  %profile  
%prun  %psearch  %psource  %pushd  %pwd  %pycat  %pylab  %quickref  %recall  %rehashx  
%reload_ext  %rep  %rerun  %reset  %reset_selective  %run  %save  %sc  %stop  %store  %sx  %tb
%time  %timeit  %unalias  %unload_ext  %who  %who_ls  %whos  %xdel  %xmode

Automagic is ON, % prefix NOT needed for magic functions.
\end{verbatim}
Note how the example above omitted the eplicit \texttt{\%} marker and
simply uses \texttt{lsmagic}. As long as the `automagic' feature is on
(which it is by default), you can omit the \texttt{\%} marker as long as
there is no ambiguity with a Python variable of the same name.

\textbf{Running your code}

While it's easy to type a few lines of code in IPython, for any
long-lived work you should keep your codes in Python scripts (or in
IPython notebooks, see below). Consider that you have a script, in this
case trivially simple for the sake of brevity, named \texttt{simple.py}:

\begin{verbatim}
In [12]: !cat simple.py
import numpy as np

x = np.random.normal(size=100)

print 'First elment of x:', x[0]
\end{verbatim}
The typical workflow with IPython is to use the \texttt{\%run} magic to
execute your script (you can omit the .py extension if you want). When
you run it, the script will execute just as if it had been run at the
system prompt with \texttt{python simple.py} (though since modules don't
get re-executed on new imports by Python, all system initialization is
essentially free, which can have a significant run time impact in some
cases):

\begin{verbatim}
In [13]: run simple
First elment of x: -1.55872256289
\end{verbatim}
Once it completes, all variables defined in it become available for you
to use interactively:

\begin{verbatim}
In [14]: x.shape
Out[14]: (100,)
\end{verbatim}
This allows you to plot data, try out ideas, etc, in a
\texttt{\%run}/interact/edit cycle that can be very productive. As you
start understanding your problem better you can refine your script
further, incrementally improving it based on the work you do at the
IPython prompt. At any point you can use the \texttt{\%hist} magic to
print out your history without prompts, so that you can copy useful
fragments back into the script.

By default, \texttt{\%run} executes scripts in a completely empty
namespace, to better mimic how they would execute at the system prompt
with plain Python. But if you use the \texttt{-i} flag, the script will
also see your interactively defined variables. This lets you edit in a
script larger amounts of code that still behave as if you had typed them
at the IPython prompt.

You can also get a summary of the time taken by your script with the
\texttt{-t} flag; consider a different script \texttt{randsvd.py} that
takes a bit longer to run:

\begin{verbatim}
In [21]: run -t randsvd.py

IPython CPU timings (estimated):
  User   :       0.38 s.
  System :       0.04 s.
Wall time:       0.34 s.
\end{verbatim}
\texttt{User} is the time spent by the computer executing your code,
while \texttt{System} is the time the operating system had to work on
your behalf, doing things like memory allocation that are needed by your
code but that you didn't explicitly program and that happen inside the
kernel. The \texttt{Wall time} is the time on a `clock on the wall'
between the start and end of your program.

If \texttt{Wall \textgreater{} User+System}, your code is most likely
waiting idle for certain periods. That could be waiting for data to
arrive from a remote source or perhaps because the operating system has
to swap large amounts of virtual memory. If you know that your code
doesn't explicitly wait for remote data to arrive, you should
investigate further to identify possible ways of improving the
performance profile.

If you only want to time how long a single statement takes, you don't
need to put it into a script as you can use the \texttt{\%timeit} magic,
which uses Python's \texttt{timeit} module to very carefully measure
timig data; \texttt{timeit} can measure even short statements that
execute extremely fast:

\begin{verbatim}
In [27]: %timeit a=1
10000000 loops, best of 3: 23 ns per loop
\end{verbatim}
and for code that runs longer, it automatically adjusts so the overall
measurement doesn't take too long:

\begin{verbatim}
In [28]: %timeit np.linalg.svd(x)
1 loops, best of 3: 310 ms per loop
\end{verbatim}
The \texttt{\%run} magic still has more options for debugging and
profiling data; you should read its documentation for many useful
details (as always, just type \texttt{\%run?}).

\subsection{The graphical Qt console}
If you type at the system prompt (see the IPython website for
installation details, as this requires some additional libraries):

\begin{verbatim}
$ ipython qtconsole
\end{verbatim}
instead of opening in a terminal as before, IPython will start a
graphical console that at first sight appears just like a terminal, but
which is in fact much more capable than a text-only terminal. This is a
specialized terminal designed for interactive scientific work, and it
supports full multi-line editing with color highlighting and graphical
calltips for functions, it can keep multiple IPython sessions open
simultaneously in tabs, and when scripts run it can display the figures
inline directly in the work area.

% This cell is for the pdflatex output only
\begin{figure}[htbp]
\centering
\includegraphics[width=3in]{ipython_qtconsole2.png}
\caption{The IPython Qt console: a lightweight terminal for scientific exploration, with code, results and graphics in a soingle environment.}
\end{figure}
The Qt console accepts the same \texttt{-{}-pylab} startup flags as the
terminal, but you can additionally supply the value
\texttt{-{}-pylab inline}, which enables the support for inline graphics
shown in the figure. This is ideal for keeping all the code and figures
in the same session, given that the console can save the output of your
entire session to HTML or PDF.

Since the Qt console makes it far more convenient than the terminal to
edit blocks of code with multiple lines, in this environment it's worth
knowing about the \texttt{\%loadpy} magic function. \texttt{\%loadpy}
takes a path to a local file or remote URL, fetches its contents, and
puts it in the work area for you to further edit and execute. It can be
an extremely fast and convenient way of loading code from local disk or
remote examples from sites such as the
\href{http://matplotlib.sourceforge.net/gallery.html}{Matplotlib
gallery}.

Other than its enhanced capabilities for code and graphics, all of the
features of IPython we've explained before remain functional in this
graphical console.

\subsection{The IPython Notebook}
The third way to interact with IPython, in addition to the terminal and
graphical Qt console, is a powerful web interface called the ``IPython
Notebook''. If you run at the system console (you can omit the
\texttt{pylab} flags if you don't need plotting support):

\begin{verbatim}
$ ipython notebook --pylab inline
\end{verbatim}
IPython will start a process that runs a web server in your local
machine and to which a web browser can connect. The Notebook is a
workspace that lets you execute code in blocks called `cells' and
displays any results and figures, but which can also contain arbitrary
text (including LaTeX-formatted mathematical expressions) and any rich
media that a modern web browser is capable of displaying.

% This cell is for the pdflatex output only
\begin{figure}[htbp]
\centering
\includegraphics[width=3in]{ipython-notebook-specgram-2.png}
\caption{The IPython Notebook: text, equations, code, results, graphics and other multimedia in an open format for scientific exploration and collaboration}
\end{figure}
In fact, this document was written as a Notebook, and only exported to
LaTeX for printing. Inside of each cell, all the features of IPython
that we have discussed before remain functional, since ultimately this
web client is communicating with the same IPython code that runs in the
terminal. But this interface is a much more rich and powerful
environment for maintaining long-term ``live and executable'' scientific
documents.

Notebook environments have existed in commercial systems like
Mathematica(TM) and Maple(TM) for a long time; in the open source world
the \href{http://sagemath.org}{Sage} project blazed this particular
trail starting in 2006, and now we bring all the features that have made
IPython such a widely used tool to a Notebook model.

Since the Notebook runs as a web application, it is possible to
configure it for remote access, letting you run your computations on a
persistent server close to your data, which you can then access remotely
from any browser-equipped computer. We encourage you to read the
extensive documentation provided by the IPython project for details on
how to do this and many more features of the notebook.

Finally, as we said earlier, IPython also has a high-level and easy to
use set of libraries for parallel computing, that let you control
(interactively if desired) not just one IPython but an entire cluster of
`IPython engines'. Unfortunately a detailed discussion of these tools is
beyond the scope of this text, but should you need to parallelize your
analysis codes, a quick read of the tutorials and examples provided at
the IPython site may prove fruitful.

\end{document}
